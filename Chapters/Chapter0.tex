% Chapter Template

\chapter{General introduction} % Main chapter title
\label{Chapter0} % Change X to a consecutive number; for referencing this chapter elsewhere, use \ref{ChapterX}

Plant breeding is a process that started as early as agriculture itself around 10,000
BC. Even after twelve millennia many of its aspects are still obscured behind complex
genetics and our lack to thoroughly comprehend them. But never were the challenges imposed
on breeding as vast as today. In the year 2050 the world's agriculture will be responsible
for feeding nine to ten billion people \cite{gerland2014world}, next to other inflicted
responsibilities of replacing fossil fuels with regenerate energy from plants, providing
fibers for industrial textile production and pharmaceutical applications. Those demands will be met without continuing advances in breeding and in quantitative genetics.  \\
According to \cite{wallace2018road} the history of breeding can be divided into four main
epochs that always utilized the technologies available to them in a specific era. In the
beginning breeding did not exist as a succinct field of science and was accomplished by
simple phenotypic selection by local farmers, which lead to dramatic changes
in approximately 7000 cultivated crop species compared to their wild ancestors \cite{khoury2016origins}.\\
The next era in plant breeding was sparked by the upcoming of new statistical methods and
the rediscovery of Mendelian genetics in the late 19th and early 20th century, which in
combination let to the development of quantitative genetics
\cite{tschermak1900kunstliche}; \cite{fisher1919xv}; \cite{fisher1923};
\cite{falconer1996}. Along with it came the discovery of
inbreeding and inbreeding depression, schematic design of field trials, the concept of variance component analysis, hybrid breeding and others. \\
The third stage, the genomic era of plant breeding, began with the discovery of
possibilities to asses polymorphisms in the genomes, leading up to marker-assisted
selection, linkage and QTL mapping. As marker arrays grew larger and sequencing costs
declined dramatically, those methods were succeeded by the more sophisticated and precise whole genome regression and genome-wide association studies (GWAS) with high-density marker maps \cite{hayes2001,korte2013advantages}. \\
Those technological advances allowed plant breeders to provide farmers with cultivars,
which were able to feed the exponentially growing world's populations since the
1950s. However, like any century before, the 21th imposes great challenges on
humankind. Climate change leads to different stresses in the environments and plant
breeders need to adapt to the specific requirements on high-yielding cultivars, maybe
quicker than ever before, as droughts and flooding occur more often around the world each
year. \\
This, however, is complicated because of the 7000 cultivated plants in agricultural
history only a few provide the major source of food today on a global scale, with most
important being maize (\textit{Zea mays}), wheat \textit{Triticum aestivum} and rice
\textit{Oryza sativa}.  Furthermore, during the course of breeding the elite cultivars
have lost the majority of the genetic diversity of its ancestral wild populations
\cite{walsh2018}. To continuously adapt and improve crop plants all the methods of
quantitative genetics, genomics and genome editing need to be combined in the modern age of Breeding 4.0 \cite{wallace2018road}. \\
Quantitative genetics is a multi-step process and requires high quality data of both
genomes and traits. Figure \ref{fig:quan_flow1} shows a flow diagram with the major
processes involved in going from genome assemblies to neural network aided genomic
prediction of complex traits for plant breeding and GWAS.
 

\begin{figure}[H]
  \begin{center}
    \begin{tikzpicture}[node distance=2cm, scale=0.8, transform shape]
      \node (start0) [startstop] {Selection of suitable candidates};
      \node (start) [startstop,below of=start0] {DNA extraction};
      \draw [arrow] (start0) -- (start);
      \node (seq) [process, below of=start, xshift=-3cm] {Sequencing} ;
      \draw [arrow] (start) -- (seq);
      \node (SNP) [process, below of=start, xshift=3cm, yshift=-2cm] {SNP array} ;
      \draw [arrow] (start) -- (SNP);
      \node (ga) [process, below of=seq] {Genome assembly} ;
      \draw [arrow] (seq) -- (ga);
      \node (snpca) [process, below of=ga] {Alignment \& SNP calling} ;
      \draw [arrow] (ga) -- (snpca);
      \node (imp) [io, below of=snpca, xshift=3cm] {Imputation of missing values};
      \draw [arrow] (snpca) -- (imp) ;
      \draw [arrow] (SNP) -- (imp) ;
      \node (LD) [io, below of=imp] {LD pruning} ;
      \draw [arrow] (imp) -- (LD) ;
      \node (MAF) [io, below of=LD] {MAF filtering} ;
      \draw [arrow] (LD) -- (MAF) ;
      \node (bm) [startstop, below of=MAF] {Numeric marker matrix} ;
      \draw [arrow] (MAF) -- (bm) ;
    \end{tikzpicture}
    \caption[Schematic process of genotyping for quantitative genetics]{Schematic process of genotyping for quantitative genetics analyses with its crucial steps} \label{fig:quan_flow1}
  \end{center}     
\end{figure}

Chapter \ref{Chapter1} will focus on the first parts of figure \ref{fig:quan_flow1}: the
assembly of genomes, which will be exemplified on plastid genomes and will elucidate some
of the major obstacles presented, when starting with raw DNA reads and advancing towards
genomic data, which is suitable for quantitative methods like GWAS and genomic
selection. Section \ref{Chapter2} will introduce a novel tool to perform large-scale GWAS
using modern software and computing resources, which allows to perform those analyses on large scales in reasonable amounts of time. \\
Chapter \ref{Chapter3} will give in depth introductions to machine learning and the
complex architecture of quantitative traits and brings them in the context of plant
breeding and will further elucidate how those techniques can be used to continue improving
plant germplasms for modern agriculture via genomic selection. Genomic selection is a
process during which plants are not only selected and assessed based on their phenotypic
appearances, single markers or pedigree relatedness
to other individuals, but mainly on their genomic features \cite{hayes2001}. \\
In the final chapter figure \ref{fig:quan_flow1} will be recapitulated and the main
obstacles in the process will be thoroughly decomposed and explained how this study aids
towards providing solutions for some of them.





%%% Local Variables:
%%% mode: latex
%%% TeX-master: "../main"
%%% End:
