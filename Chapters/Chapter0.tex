% Chapter Template

\chapter{General introduction} % Main chapter title
\label{Chapter0} % Change X to a consecutive number; for referencing this chapter elsewhere, use \ref{ChapterX}


Plant breeding is a process that started as early as agriculture itself around 10,000
BC. Even after twelve millennia many of its aspects are still obscured behind complex
genetics and our lack to thoroughly comprehend them. But never were the challenges the
vast as today. In the year 2050 this world's agriculture will be responsible for feeding 9
to 10 billion people \cite{gerland2014world}, next to other inflicted duties of replacing
fossil fuels with regenerate energy from plants, providing fibers for industrial textile
production and pharmaceutical applications. \\
According to \cite{wallace2018road} the history of breeding can be divided into four main
epochs that always utilized to technologies available to them in a specific era. The
majority of this time breeding, as a succinct field of science, did not exist and breeding
was executed by simple phenotypic selection by local farmers leading to dramatic changes
in ca. 7000 cultivated crop species \cite{khoury2016origins}. The next era in plant
breeding was sparked by the upcoming of new statistical methods and the rediscovery of
Mendelian genetics in the late 19th and early 20th century, which combined let the
development of quantitative genetics \cite{tschermak1900kunstliche}; \cite{fisher1919xv};
\cite{fisher1923}; \cite{falconer1996}. With it came the discovery of inbreeding and
inbreeding depression, schematic design of field trials, the concept of variance component analysis, hybrid breeding and others. \\
The third stage, the genomic era of plant breeding, began with the discovery of molecular
markers, leading up to marker-assisted breeding, linkage and QTL mapping. As marker arrays
grew larger and sequencing costs dramatically reduced, they were succeeded by the more
sophisticated and precise methods of whole genome regression and genome-wide association
studies (GWAS) with high-density marker maps \cite{hayes2001,korte2013advantages}. \\
Those technological advances allowed plant breeders to provide farmers with cultivars,
which were able to feed the exponentially growing world's populations since the
1950s. However, like any century before the 21th imposes great challenges on
humankind. Climate change leads to different stresses in the environments and plant
breeders need to adapt to the specific requirements on high-yielding cultivars, maybe
quicker than ever before, as drought and flooding occur more often around the world each
year. \\
Artificial selection has similar effects as bottlenecks in natural selection do on
populations.  Of the 7000 cultivated plants in agricultural history, today only three
provide the major source of food around the world. All three of them -- maize (\textit{Zea
  mays}) , wheat \textit{Triticum aestivum} and rice \textit{Oryza sativa} -- are closely
related from the family of \textit{poaceae}. Furthermore during the course of breeding the
elite cultivars have lost the majority of the genetic diversity of its ancestral wild
populations \cite{walsh2018}. To continuously adapt and improve crop plants all the methods of quantitative genetics, genomics and genome editing need to be combined in the modern age of Breeding 4.0 \cite{wallace2018road}
