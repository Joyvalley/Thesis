% Chapter Template

\chapter{General Introduction} % Main chapter title
\label{Chapter0} % Change X to a consecutive number; for referencing this chapter elsewhere, use \ref{ChapterX}

Plant breeding is a process that started as early as agriculture itself around 10,000
BC. Even after twelve millennia many of its aspects are still obscured behind complex
genetics and our lack to thoroughly comprehend them. But never were the challenges the
vast as today. In the year 2050 the world's agriculture will be responsible for feeding
nine to ten billion people \cite{gerland2014world}, next to other inflicted duties of
replacing fossil fuels with regenerate energy from plants, providing fibers for industrial
textile production and pharmaceutical applications. \\
According to \cite{wallace2018road} the history of breeding can be divided into four main
epochs that always utilized the technologies available to them in a specific era. In the
beginning breeding did not exist as succinct field of science and was accomplished by
simple phenotypic selection by local farmers leading to dramatic changes
in ca. 7000 cultivated crop species \cite{khoury2016origins}.\\
The next era in plant breeding was sparked by the upcoming of new statistical methods and
the rediscovery of Mendelian genetics in the late 19th and early 20th century, which
combined let to the development of quantitative genetics \cite{tschermak1900kunstliche};
\cite{fisher1919xv}; \cite{fisher1923}; \cite{falconer1996}. With it came the discovery of
inbreeding and inbreeding depression, schematic design of field trials, the concept of variance component analysis, hybrid breeding and others. \\
The third stage, the genomic era of plant breeding, began with the discovery of
possibilities to asses polymorphisms in the genomes, leading up to marker-assisted
breeding, linkage and QTL mapping. As marker arrays grew larger and sequencing costs
dramatically reduced, they were succeeded by the more sophisticated and precise methods of
whole genome regression and genome-wide association
studies (GWAS) with high-density marker maps \cite{hayes2001,korte2013advantages}. \\
Those technological advances allowed plant breeders to provide farmers with cultivars,
which were able to feed the exponentially growing world's populations since the
1950s. However, like any century before, the 21th imposes great challenges on
humankind. Climate change leads to different stresses in the environments and plant
breeders need to adapt to the specific requirements on high-yielding cultivars, maybe
quicker than ever before, as drought and flooding occur more often around the world each
year. \\
Artificial selection has similar effects as bottlenecks in natural selection do on
populations. Of the 7000 cultivated plants in agricultural history, today only few provide
the major source of food around the world, with most important being maize (\textit{Zea
  mays}) , wheat \textit{Triticum aestivum} and rice \textit{Oryza sativa}.  Furthermore,
during the course of breeding the elite cultivars have lost the majority of the genetic
diversity of its ancestral wild populations \cite{walsh2018}. To continuously adapt and
improve crop plants all the methods of quantitative genetics, genomics and genome editing
need to be combined in the modern age of Breeding 4.0 \cite{wallace2018road}. \\
Quantitative genetics is a multi step process and requires high quality data have both
genomes and traits. Figure \ref{fig:quan_flow1} shows a flow diagram with the major
processes involved in going from genome assemblies to neural network aided genomic
prediction of complex traits for plant breeding. 


\begin{figure}[H]
  \begin{center}
    \begin{tikzpicture}[node distance=2cm, scale=0.8, transform shape]
      \node (start0) [startstop] {Selection of suitable candidates};
      \node (start) [startstop,below of=start0] {DNA extraction};
      \draw [arrow] (start0) -- (start);
      \node (seq) [process, below of=start, xshift=-3cm] {Sequencing} ;
      \draw [arrow] (start) -- (seq);
      \node (SNP) [process, below of=start, xshift=3cm, yshift=-2cm] {SNP array} ;
      \draw [arrow] (start) -- (SNP);
      \node (ga) [process, below of=seq] {Genome assembly} ;
      \draw [arrow] (seq) -- (ga);
      \node (snpca) [process, below of=ga] {Alignment \& SNP calling} ;
      \draw [arrow] (ga) -- (snpca);
      \node (imp) [io, below of=snpca, xshift=3cm] {Imputation of missing values};
      \draw [arrow] (snpca) -- (imp) ;
      \draw [arrow] (SNP) -- (imp) ;
      \node (LD) [io, below of=imp] {LD pruning} ;
      \draw [arrow] (imp) -- (LD) ;
      \node (MAF) [io, below of=LD] {MAF filtering} ;
      \draw [arrow] (LD) -- (MAF) ;
      \node (bm) [startstop, below of=MAF] {Numeric marker matrix} ;
      \draw [arrow] (MAF) -- (bm) ;
    \end{tikzpicture}
    \caption[Schematic process of genotyping for quantitative genetics]{Schematic process of genotyping for quantitative genetics analyses with its crucial steps} \label{fig:quan_flow1}
  \end{center}     
\end{figure}

Chapter \ref{Chapter1} will focus on the first parts of figure \ref{fig:quan_flow1}, the
assembly of genomes exemplified with plastid genomes and will elucidate some of the major
obstacles presented, when starting with raw DNA reads and going to data, which is suitable
for quantitative methods like GWAS and genomic selection. Section \ref{Chapter2} will
introduce a novel tool to perform large-scale GWAS using modern software and computing
resources. \\
Chapter \ref{Chapter3} will give in depth introductions to machine learning and the
complex architecture of quantitative traits and bring them in the context of plant
breeding and how those techniques can be used to continue improving plant germplasms for
modern agriculture. Genomic selection is a process during which plants are not only
selected and assessed based on their bare phenotypic appearances or pedigree relatedness
to other individuals but also based on their genomic features \cite{hayes2001}. \\
In the final chapter figure \ref{fig:quan_flow1} will be recapitulated and the main
obstacles in the process will be thoroughly decomposed and explained how this study aids
in providing solutions for some of them. 





%%% Local Variables:
%%% mode: latex
%%% TeX-master: "../main"
%%% End:
