% Chapter 1



\chapter{Benchmarking of Chloroplast Genome Assembly tools } % Main chapter title

\label{Chapter1} % For referencing the chapter elsewhere, use \ref{Chapter1}
This chapter orientates on \cite{freudenthal2019landscape} only the chapters from the publication which the author majorly contributed to are included. 

%----------------------------------------------------------------------------------------

% Define some commands to keep the formatting separated from the content 
\newcommand{\keyword}[1]{\textbf{#1}}
\newcommand{\tabhead}[1]{\textbf{#1}}
\newcommand{\code}[1]{\texttt{#1}}
\newcommand{\file}[1]{\texttt{\bfseries#1}}
\newcommand{\option}[1]{\texttt{\itshape#1}}

%----------------------------------------------------------------------------------------

\section{Introduction}

Circular DNA of a size between 120 kBP to 160 kBP \cite{palmer_1985}.  First chloroplast sequenced as early
as 1986 \textit{Marchantia polymorpha} and \textit{Nico} \cite{ohyama_chloroplast_1986}; \cite{shinozaki_complete_1986}.
Review  genome structure \cite{green_chloroplast_2011}; \cite{wicke_evolution_2011}.
Chloroplast genomes widely used in evolutionary studies \cite{martin_evolutionary_2002}; \cite{xiao-ming_inferring_2017}.
Chloroplast genomes are small through endosymbiotic gene transfer \cite{martin_evolutionary_2002}; \cite{deiner_environmental_2017}. Up to 14 \% of the the core genome of \textit{Arabidopsis thaliana} is made up of genes previously from the chloroplast (fancy citation), while 100-120 genes remain on the chloroplast \cite{wicke_evolution_2011}.
However chloroplast between plant species show large variety e.g. parasitic plants.
Chloroplast genomes are much smaller than core genomes e.g \textit{A. thaliana} ca. 125 MBP. Highly conserved high gene content, therefore changes are more likely to be functional.
Single chloroplast contain up to hundreds of copies of its genome \cite{kumar_2014}; \cite{bendich_1987}, and photosynthetic activate cells contain multiple chloroplasts, therefore the copy number of chloroplast genomes is much higher than number of core genomes per cell.
Structurally chloroplast genomes consist of two inverted repeats (IR) - $IR_A$ and $IR_B$ -  ranging from 10 kbp to 76 kB the divide the chloroplast genome into two distinct regions the large single copy (LSC) and the small single copy (SSC) as shown in \ref{fig:cpast_genome} \cite{palmer_1985}. Taking into account that the majority of assembly tools has been designed to assemble linear core genomes, the structure of chloroplast genomes is a major obstacle when wanting to assemble those genomes with modern short read technologies, especially solving and aligning the IR \cite{Wang2018}.

\begin{figure}[H]
\centering
\includegraphics[height=.55\textheight, width=.95\textwidth]{Figures/cpast}
\decoRule
\caption[Structure of a chloroplast genome]{Structure of the chloroplast genome of \textit{A. thaliana} with SSC-Region and LSC-Region. Length of the genome and its parts in kbp. Graphic from \cite{olejniczak2016chloroplasts}
\label{fig:cpast_genome}
\end{figure}




%----------------------------------------------------------------------------------------
\section{Material and Methods}
\subsection{Methods}
\subsection{Tools}
\subsection{Evaluation}
\subsubsection{Quantitative}
\begin{equation}
  score = \frac{1}{4} \cdot \left( cov_{ref} +  cov_{qry} + min\left\{ \frac{cov_{qry}}{cov_{ref}}, \frac{cov_{ref}}{cov_{qry}}\right\} + \frac{1}{n_{contiguous} }\right) \cdot 100 
\label{eqn:score_ass}
\end{equation}

\subsubsection{Qualitative}
\subsubsection{Consistency}


\subsection{Data}
\subsubsection{Simulated}
\subsubsection{Real data set}
\subsubsection{Novel data set}


\section{Results}
\subsection{Qualitative}

\subsection{Quantitative}
\subsubsection{Simulated data}
\begin{figure}[H]
\centering
\includegraphics[height=.45\textheight, width=.95\textwidth]{Figures/sim_tiles}
\decoRule
\caption[Score of assemblies of simulated data sets]{Results of assemblies executed with simulated data sets.}
\label{fig:sim_tiles}
\end{figure}



\begin{table}[h!]
\caption{\textbf{Scores of assemblies of simulated data}}
\label{tab:scores_simulated}
\centering
\begin{tabular}{rlrrrrrrr}
  \hline
  & data set & CAP & CE & Fast-Plast & GetOrganelle & IOGA & NOVOPlasty & org.ASM \\ 
  \hline
  1 & sim\_150bp.0-1 & 79.10 & 100.00 & 99.48 & 100.00 &  & 91.52 & 100.00 \\ 
  2 & sim\_150bp.0-1.2M & 79.10 & 100.00 & 99.72 & 100.00 & 79.10 & 91.52 & 91.50 \\ 
  3 & sim\_150bp.1-10 &  & 56.44 & 100.00 & 76.98 &  & 91.52 & 78.00 \\ 
  4 & sim\_150bp.1-10.2M &  &  & 99.97 & 100.00 &  & 91.52 & 82.72 \\ 
  5 & sim\_150bp.1-100 & 75.72 & 100.00 & 99.48 & 100.00 & 66.09 & 91.52 & 91.50 \\ 
  6 & sim\_150bp.1-100.2M &  & 100.00 & 99.47 & 100.00 &  & 100.00 & 100.00 \\ 
  7 & sim\_150bp.1-1000 & 79.10 &  & 99.72 & 100.00 &  & 91.52 & 100.00 \\ 
  8 & sim\_150bp.1-1000.2M & 79.10 & 100.00 & 99.72 & 100.00 &  & 91.52 & 100.00 \\ 
  9 & sim\_250bp.0-1 & 79.10 & 100.00 & 93.82 & 100.00 &  & 91.52 & 91.50 \\ 
  10 & sim\_250bp.0-1.2M & 79.10 & 100.00 & 93.83 & 100.00 &  & 91.52 & 91.50 \\ 
  11 & sim\_250bp.1-10 &  & 54.98 & 68.45 & 78.89 & 52.71 & 91.52 & 40.20 \\ 
  12 & sim\_250bp.1-10.2M &  &  & 93.00 & 100.00 & 52.67 & 87.40 & 40.20 \\ 
  13 & sim\_250bp.1-100 & 72.81 & 100.00 & 93.82 & 100.00 &  & 87.40 & 100.00 \\ 
  14 & sim\_250bp.1-100.2M &  & 100.00 & 93.83 & 100.00 &  & 87.40 & 100.00 \\ 
  15 & sim\_250bp.1-1000 & 79.10 & 21.30 & 93.83 & 100.00 & 76.96 & 91.52 & 91.50 \\ 
  16 & sim\_250bp.1-1000.2M & 79.10 & 100.00 & 93.83 & 100.00 & 67.55 & 87.40 & 100.00 \\
  \hline
\end{tabular}
\end{table}




\subsubsection{Real data sets}
\begin{table}[h!]
\caption{\textbf{Mean scores of chloroplast genome assemblers}}
\label{tab:scores_real}
\centering
\begin{tabular}{rlrrrr}
  \hline
 & assembler & Median & IQR & N\_perfect & N\_tot \\ 
  \hline
1 & CAP & 45.25 & 50.19 &   0 & 369 \\ 
  2 & CE & 56.55 & 71.50 &  14 & 369 \\ 
  3 & Fast-Plast & 92.80 & 23.59 & 113 & 369 \\ 
  4 & GetOrganelle & 99.83 & 20.94 & 210 & 360 \\ 
  5 & IOGA & 71.10 & 11.21 &  0 & 338 \\ 
  6 & NOVOPlasty & 75.95 & 48.69 &  58 & 369 \\ 
  7 & org.ASM & 67.35 & 91.69 &  46 & 348 \\ 
   \hline
\end{tabular}
\end{table}





\begin{figure}[H]
\centering
\includegraphics[height=.45\textheight, width=.95\textwidth]{Figures/swarm}
\decoRule
\caption[Scores of assemblies from real data sets]{Box and swarm plots depict the results from the scoring shown in \ref{eqn:score_ass}}
\label{fig:swarm}
\end{figure}




\subsubsection{Consistency}

\subsubsection{Real data sets}
\begin{figure}[H]
\centering
\includegraphics[height=.45\textheight, width=.95\textwidth]{Figures/repro}
\decoRule
\caption[Comparison between two runs with the same assembler for consistency testing ]{Swarm plots depict the results from the scoring shown in \ref{eqn:score_ass} for two independent runs for each assembler on each of the data sets}
\label{fig:consisplot}
\end{figure}


\subsubsection{Novel}


\begin{figure}[H]
\centering
\includegraphics[height=.45\textheight, width=.95\textwidth]{Figures/upset_novel}
\decoRule
\caption[Upset plot comparing the success rates for novel data sets]{nls}
\label{fig:upset_novel}
\end{figure}

\section{Discussion}

\begin{figure}[H]
\centering
\includegraphics[height=.45\textheight, width=.95\textwidth]{Figures/upset}
\decoRule
\caption[Upset plot comparing the success rates of of all assemblers]{Upset plot showing the intersections of success rates between assemblers. A successful assembly was defined with a score > 99 according to equation \ref{eqn:score_ass}}
\label{fig:upset}
\end{figure}


