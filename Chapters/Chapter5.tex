% Chapter 5

\chapter{GWAS}

\label{Chapter5} % For referencing the chapter elsewhere, use \ref{Chapter1} 

%----------------------------------------------------------------------------------------
\section{Introduction}
Chapter 2 or 3 introduced a new framework for high throughout GWAS. GWAS-Flow enables to execute GWAS analysis
on GPU and thus gains an incredible performance advantage compared to other state of the art GWAS
technologies. This enabled the research group of evolutionary genomics at the CCTB to evaluate 462 (oder so)
data sets from Arapheno with the full 10 mio data set originally imputed with Beagle version 3.  Those GWAS
have been repeated 100 times with shuffled phenotypes to estimate permutation based thresholds. Additionally
all phenotypes were re-evaluated with a recently generated data set which based the imputation on the more
recent Beagle version 5, which supposingly is more accurate (Pook et al). In the scope of this chapter it will
be assessed if this does influence the outcome of the GWAS. Furthermore was GWAS-Flow used to evaluate some of
the common practices in GWAS, that a prior do not seam relevant to the author. Sometimes yet uncommon in the
literature it is recommend to use transformations for the pheonotypic values prior to performing GWAS to
account for non-normal distributed data, transformations or calculated with the log, square root or
boxcox. Another common practice is to use principal components from PCA to account for population
structure. This however should be covered sufficiently by the K matrix which is part of basically any mixed
linear model based GWAS approach.


\section{Reevalulation of 463 phenotypes from the AraPheno database}

\subsection{Introduction}
\subsection{Material and Methods}
\subsection{Results}
\subsection{Results}
\subsection{Disucssion}



\section{GWAS in DH landrace populatios of maze across and within environments }

\subsection{Introduction}
\subsection{Material and Methods}
\subsection{Results}
\subsection{Results}
\subsection{Disucssion}

