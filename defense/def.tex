\documentclass{beamer}
\usepackage{graphicx}
\usepackage{float}
\usepackage{amsmath}
\usepackage{caption}
\usepackage{subcaption}
\usepackage{setspace}
\usepackage[english,ngerman]{babel}
\usepackage[T1]{fontenc}
\usepackage{tikz}
\usepackage{epigraph}
\captionsetup[figure]{labelformat=empty}
\selectlanguage{english}
\graphicspath{{../Figures/}}
\usetheme{Dresden}
\usecolortheme{spruce}
\useoutertheme{infolines}
\useinnertheme{rectangles}
\setbeamercovered{transparent} 
\colorlet{mdtRed}{red!50!black}
\beamertemplatenavigationsymbolsempty 

\usepackage{tikz}
\usetikzlibrary{shapes.geometric, arrows}
\tikzstyle{startstop} = [rectangle, rounded corners, minimum width=3cm, minimum height=1cm,text centered, draw=black, fill=red!30]
\tikzstyle{io} = [trapezium, trapezium left angle=70, trapezium right angle=110, minimum width=3cm, minimum height=1cm, text centered, draw=black, fill=blue!30]
\tikzstyle{process} = [rectangle, minimum width=3cm, minimum height=1cm, text centered, draw=black, fill=orange!30]
\tikzstyle{decision} = [diamond, minimum width=3cm, minimum height=1cm, text centered, draw=black, fill=green!30]
\tikzstyle{arrow} = [thick,->,>=stealth]




 \title[Quantitative genetics] %optional
{\color{mdtRed}{Quantitative genetics from genome assemblies to neural network aided omics-based prediction of complex traits}}
 
 
\author{Jan Freudenthal}
 
\institute[CCTB] % (optional)
{
  % 
  CCTB \\
  Evolutionary genomics \\
  Julius-Maximilians-Universität Würzburg
}
 
\date{31. Jan 2020}
 
%\logo{\includegraphics[height=1.5cm]{lion-logo.png}}

\begin{document}
\frame{\titlepage}

\begin{frame}
  \frametitle{Quantitative genetics}
  Quantitative genetics aims to explain the heritable parts of traits that follow certain statistical distributions.
\end{frame}


\begin{frame}
    \frametitle{Quantitative genetics}
    \includegraphics[width=.98\textwidth, height=.8\textheight]{../Figures/gxe.png}
\end{frame}

\begin{frame}
    \frametitle{Quantitative genetics}
    \includegraphics[width=.98\textwidth, height=.8\textheight]{../Figures/athal.png}
\end{frame}




\begin{frame}
  \begin{huge}
    \frametitle{Decomposition of phenotypic variance} \vspace{-2ex}
    \begin{itemize}[<+->]
    \item[] \[\sigma_P = \sigma_G + \sigma_E + \sigma_{GxE}\]
    \item[] \[\sigma_G = \sigma_A + \sigma_D + \sigma_I\]
    \item[] \[\sigma_I = \sigma_{AA} + \sigma_{AD} + \sigma_{DD}\]
    \item[] \[h^2 = \frac{\sigma_A}{\sigma_P} \]
    \end{itemize}
  \end{huge}
\end{frame}
 



\begin{frame}
  \frametitle{Workflow in quantitative Genetics}
  
\begin{figure}[H]
  \begin{center}
    \begin{tikzpicture}[node distance=2cm, scale=0.8, transform shape]
      \node (start0) [startstop] {Selection of suitable candidates};
      \node (start) [startstop,below of=start0] {DNA extraction};
      \draw [arrow] (start0) -- (start);
      \node (seq) [process, below of=start, xshift=-3cm] {Sequencing} ;
      \draw [arrow] (start) -- (seq);
      \node (SNP) [process, below of=start, xshift=3cm, yshift=-2cm] {SNP array} ;
      \draw [arrow] (start) -- (SNP);
      \node (ga) [process, below of=seq] {Genome assembly} ;
      \draw [arrow] (seq) -- (ga);
      \node (snpca) [process, below of=ga] {Alignment \& SNP calling} ;
      \draw [arrow] (ga) -- (snpca);
      \node (imp) [io, below of=snpca, xshift=3cm] {Imputation of missing values};
      \draw [arrow] (snpca) -- (imp) ;
      \draw [arrow] (SNP) -- (imp) ;
      \node (LD) [io, below of=imp] {LD pruning} ;
      \draw [arrow] (imp) -- (LD) ;
      \node (MAF) [io, below of=LD] {MAF filtering} ;
      \draw [arrow] (LD) -- (MAF) ;
      \node (bm) [startstop, below of=MAF] {Numeric marker matrix} ;
      \draw [arrow] (MAF) -- (bm) ;
    \end{tikzpicture}
    \caption[Schematic process of genotyping for quantitative genetics]{Schematic process of genotyping for quantitative genetics analyses with its crucial steps} \label{fig:quan_flow1}
  \end{center}     
\end{figure}
\end{frame}

\begin{frame}
\vspace{-22em}
\begin{figure}[H]
  \begin{center}
    \begin{tikzpicture}[node distance=2cm, scale=0.8, transform shape]
      \node (start0) [startstop] {Selection of suitable candidates};
      \node (start) [startstop,below of=start0] {DNA extraction};
      \draw [arrow] (start0) -- (start);
      \node (seq) [process, below of=start, xshift=-3cm] {Sequencing} ;
      \draw [arrow] (start) -- (seq);
      \node (SNP) [process, below of=start, xshift=3cm, yshift=-2cm] {SNP array} ;
      \draw [arrow] (start) -- (SNP);
      \node (ga) [process, below of=seq] {Genome assembly} ;
      \draw [arrow] (seq) -- (ga);
      \node (snpca) [process, below of=ga] {Alignment \& SNP calling} ;
      \draw [arrow] (ga) -- (snpca);
      \node (imp) [io, below of=snpca, xshift=3cm] {Imputation of missing values};
      \draw [arrow] (snpca) -- (imp) ;
      \draw [arrow] (SNP) -- (imp) ;
      \node (LD) [io, below of=imp] {LD pruning} ;
      \draw [arrow] (imp) -- (LD) ;
      \node (MAF) [io, below of=LD] {MAF filtering} ;
      \draw [arrow] (LD) -- (MAF) ;
      \node (bm) [startstop, below of=MAF] {Numeric marker matrix} ;
      \draw [arrow] (MAF) -- (bm) ;
    \end{tikzpicture}
    \caption[Schematic process of genotyping for quantitative genetics]{Schematic process of genotyping for quantitative genetics analyses with its crucial steps} \label{fig:quan_flow1}
  \end{center}     
\end{figure}
\end{frame}


\begin{frame}
  \frametitle{Numeric marker matricies}
  \includegraphics[height=.8\textheight,width=.98\textwidth]{../Figures/chr1_hap}
\end{frame}

\begin{frame}
\begin{table}[H]
 \centering
 \caption[Environmentally enhanced marker matrix]{Schematic representation of the enhanced genotype matrix for across environment prediction of maize phenotypes with DHs 1-2 with markers M 1-2 in environments E1-2}
 \label{tab:envmarker}
 \begin{tabular}{l|cccc}
   \hline
        & M-1 & M-2 & M-3 & M-4 \\
   \hline
   Acc1 & 0   & 1   & 1   & 0   \\
   Acc2 & 1   & 0   & 1   & 0   \\
   Acc3 & 0   & 1   & 0   & 1   \\
   Acc4 & 1   & 0   & 0   & 1   \\
   \hline
 \end{tabular}
\end{table}
\end{frame}


\begin{frame}
  \frametitle{Methods in quantitative genetics}
  \includegraphics[height=.8\textheight,width=.9\textwidth]{bla.pdf}
\end{frame}



 \end{document}
%%% Local Variables:
%%% mode: latex
%%% TeX-master: t
%%% End:
