% Appendix A


\chapter{Genomic prediction} % Main appendix title

\label{AppendixC} % For referencing this appendix elsewhere, use \ref{AppendixA}

\section{GP ANN}
\begin{lstlisting}[language=Python]

  import os,sys,gc
import pandas as pd
import numpy as np
import timeit
from datetime import datetime
import keras
import tensorflow as tf
from keras import backend as K
from keras import layers
from keras.models import Sequential
from keras.layers import Dense, Dropout, GaussianNoise, AlphaDropout, Reshape
from keras.layers import Flatten, LocallyConnected1D, LocallyConnected2D
from keras.optimizers import Adam, Adagrad, Adadelta
from keras.backend.tensorflow_backend import set_session 

##set default values

learning_rate = 0.01
JobID = 1
ps = 25
optim = "adam"
X_file = "KE.geno.csv"
Y_file = "KE_pheno.csv"
CV_file = "KE_cv_pw.csv"
label = "DtSILK"
start_time = timeit.default_timer()
act="relu"
drop_rate = str('0.5,0.5,0.5')
arc = str('63,63')
DG = 'D,D,D,D,D,G'
LC = True
training_epochs = 25
hyp = False

### parse command line arguments

for i in range (1,len(sys.argv),2):
    if sys.argv[i] == "-x":
        X_file = sys.argv[i+1]
    elif sys.argv[i] == "-y":
        Y_file = sys.argv[i+1]
    elif sys.argv[i] == "-cv":
        CV_file = sys.argv[i+1]
    elif sys.argv[i] == "-JobID":
        JobID = int(sys.argv[i+1])
    elif sys.argv[i] == "-label":
        label = sys.argv[i+1]
    elif sys.argv[i] == "-act":
        act = str(sys.argv[i+1])
    elif sys.argv[i] == "-epochs":
        training_epochs = int(sys.argv[i+1])
    elif sys.argv[i] == "-lr":
        learning_rate = float(sys.argv[i+1])
    elif sys.argv[i] == "-arc":
        arc = sys.argv[i+1]
    elif sys.argv[i] == "-ps":
        ps = int(sys.argv[i+1])
    elif sys.argv[i] == "-dr":
        drop_rate=str(sys.argv[i+1])
    elif sys.argv[i] == "-LC":
         LC = bool(sys.argv[i+1])
    elif sys.argv[i] == "hyp":
        hyp = bool(sys.argv[i+1])
    else:
        print('unknown option ' + str(sys.argv[i]))
        quit()

        
        
## change dir to data location


#os.chdir('/home/jaf81qa/jan_storage/tens')
x = pd.read_csv(X_file, index_col = 0)
#os.chdir("/storage/full-share/genoPred/maze")
y = pd.read_csv(Y_file, index_col = 0)
cv_folds = pd.read_csv(CV_file,index_col=0)

## select column of phenotype file via columnname

y = y[[label]]
## activity_regularizer=regularizers.l1(0.01)))

def build_network(arc,drop_rate,LC,DG):
    def add_drops(model,drop_out,k):
        if DG[k].upper() == 'D':
            model.add(Dropout(drop_out[0]))
        elif DG[k].upper() == 'G':
            model.add(GaussianNoise(drop_out[k]))
        elif DG[k].upper() == "A":
            model.add(AlphaDropout(drop_out[k]))
        else:
            pass
        return model    
    DG = DG.strip().split(",")
    arc = arc.strip().split(",")
    archit = []
    for layer in  arc:
        archit.append(int(layer))
    layer_number = len(archit)        
    drop_rate = drop_rate.strip().split(",")
    drop_out = []
    for drops in drop_rate:
        drop_out.append(float(drops)) 
    model = Sequential()
    if LC == True:
        model.add(Reshape(input_shape=(x_train.shape[1],),target_shape=(x_train.shape[1],1)))
        model.add(LocallyConnected1D(1,10,strides=7,input_shape=(x_train.shape[1],1)))
        model.add(Flatten())
        start = 0
        model = add_drops(model,drop_out,start)
    elif LC == False:
        model.add(Dense(archit[0], kernel_initializer='truncated_normal', activation=act, input_shape=(x_train.shape[1],)))
        model = add_drops(model,drop_out,start)
    start = 1
    for k in range(start,len(archit)):
        model.add(Dense(archit[k], kernel_initializer='truncated_normal', activation=act))
        model = add_drops(model,drop_out,k)
    model.add(Dense(1, kernel_initializer='truncated_normal'))
    return(model)

     
config = tf.ConfigProto()
#config.gpu_options.per_process_gpu_memory_fraction = 0.1
config.gpu_options.allow_growth = True
set_session(tf.Session(config=config))


if not os.path.isfile("RESULTScv50.txt"):
    out2 = open("RESULTScv50.txt",'w')
    out2.write('DateTime\tCompTime\tDF\tGenos\tPhenos\tCV_fold\tArchit\tConv\tActFun\tEpochs\tdrop_rate\tAccuracy\n' )

    
for k in range(1,51):
    print("Training on cv fold "+ str(k))
    cv = cv_folds['cv_' + str(k)]
    num_cvs = np.ptp(cv) + 1
    
    i = 1
    x_train = x[cv != i] 
    x_test = x[cv == i] 
    y_train = y[cv != i]
    y_test = y[cv == i]

    yhat = np.zeros(shape = y_test.shape)

    model = build_network(arc,drop_rate,LC,DG)
    model.compile(loss='mse', optimizer=Adam(lr=0.01,decay = 0.001),metrics=['accuracy'])
    model.fit(x_train,y_train, epochs=training_epochs , verbose=0) 
#    score = model.evaluate(x_test, y_test, verbose=0)
    bla = model.predict(x_test)
    y_sub= y[np.asarray(cv == i)]
    
    print(model.summary())
    print('\n')
    print(label)        

    comp_time = int(round(timeit.default_timer() - start_time,0))

    DateTime = datetime.now().strftime('%Y-%m-%d %H:%M:%S')
    acc = np.corrcoef(bla[:,0],np.asarray(y_sub)[:,0])[0,1]

    out2 = open("RESULTScv50.txt", 'a')
    out2.write('%s\t%i\t%s\t%s\t%s\t%i\t%s\t%s\t%s\t%i\t%s\t%0.5f\n' % (
        DateTime, comp_time, label, X_file, Y_file, int(k), arc, LC,  act,int(training_epochs), drop_rate, round(acc,4)))

    del model,bla, x_train, x_test, y_train, y_test 
    K.clear_session() 
    gc.collect()
    
    config = tf.ConfigProto()
    #config.gpu_options.per_process_gpu_memory_fraction = 0.1
    config.gpu_options.allow_growth = True
    set_session(tf.Session(config=config))
\end{lstlisting}

\section{GBLUP script}
\begin{lstlisting}[language=R]
    geno_pred <- function(phenocsv,genocsv,cvfcsv,cvf=1,mod = "BRR",label,phe)
{
    my_phe <- phe
    depends<- c("BGLR","doBy","doParallel",'R.utils',"BBmisc","dplyr")
    foo <- sapply(depends,
                  function(X){if(!suppressPackageStartupMessages(require(X,character.only = T))){install.packages(X)}})
    foo <- sapply(depends,function(X){suppressPackageStartupMessages(library(X,character.only=TRUE))})
    rm(foo)
    
        
    maze <- read.csv(genocsv, row.names = 1)
    phe <- read.csv(phenocsv, row.names = 1)
    cvffolds <- read.csv(cvfcsv,row.names=1)
    
    X <- scale(maze)
    y <- phe[[label]]
    if(any(is.na(y))){
        rms <- which(is.na(y))
        y <- y[-rms]
        X <- X[-rms,]
    }
    for(i in 1:50){
        cvf = i
        n=length(y)
        seed <- sample(1:100,1)
                                        #set.seed(seed)
                                        #folds=sample(1:cvf,size=n,replace=T)
        folds = cvffolds[,cvf]
        yHatCV=rep(NA,n)
        
        for(i in 1:max(folds)){
            cat("Predicting cv-fold ",i," of ", max(folds))
            tst=which(folds==i)
            yNA=y
            yNA[tst]=NA
            fm=BGLR(y=yNA,ETA=list(list(X=X,model=mod)),verbose =F ,nIter=7000,burnIn=1000)
            yHatCV[tst]=fm$yHat[tst]
            cat("   done\n")
        }
        
        my_cor <- cor(yHatCV,y,use = "complete.obs")
        print(c("Corrleation of GP", mod, my_cor))
        filename = paste0(my_phe,"_gp_results.csv")
        print(filename)
        if(!any(dir() == filename)){
            res <- matrix(ncol=8, nrow = 1) %>%
                setColNames(c("geno","pheno","cv_folds","seed","label", "cor","method","nmark"))
            res[1,] <- c(as.character(genocsv),as.character(phenocsv),as.character(cvf),as.character(seed),
                         as.character(label),as.character(my_cor),as.character(mod),dim(X)[2])
            print("#################")
	    print(res)
	    print("############")
            write.csv(res,filename)
        }else{
            res <- read.csv(filename,row.names = 1)
            for(i in 1:7){
                res[,i] <- as.character(res[,i])
            }
            res[dim(res)[1]+1,] <- c(as.character(genocsv),phenocsv,cvf,seed,as.character(label),my_cor,mod,dim(X)[2])
            write.csv(res,filename)
        }
    }
    
  }
## execute this script with: Rscript ex.gblup.r -x genofile -y phenofile -c cv file
source("~/PHD/Projects/gblup/bglr.r")


my.args <- commandArgs(trailingOnly = TRUE)
#my.args <- c("-x", "gent_geno.csv","-y" , "gent_pheno.csv")
### set defaults 
#cvf.name = NA

## parsing the command line options 
all.opts <- c("-x","-y","-label","-h","-cv","-phe")
for(i in 1:length(my.args)){
    if( i %% 2 == 1){
        if(!my.args[i] %in%  all.opts){
            cat("unknown option", my.args[i], "Use only", all.opts , "\n")
            cat("use -h for help \n")
            quit()
        }
    }    
    if(my.args[i] == "-x"){
        geno.name <- as.character(my.args[i+1])
    } else if(my.args[i] == "-y") {
        pheno.name <- as.character(my.args[i+1])
    } else if(my.args[i] == "-label") {
        my_ph <- as.character(my.args[i+1])
    } else if(my.args[i] == "-cv"){
        cv.name = as.character(my.args[i+1])
    } else if(my.args[i] == "-phe"){
        my_phe <- as.character(my.args[i+1])
    } else if(my.args[i] == "-h") {
        print(" This script takes as a minimum two intputs\n")
        print(" -x genotypefile")
        print(" -y phenotypefile ")
        print(" -cv cross-valiadtion file : is optional if none is specified random 5 fold cv will be used")
        print(" -JobID : specify column number to use in your cross validation file")
        print(" -label : use header of phenotype file column you want to use")
        quit()
    }
}


#pheno.name <- my.args[1]
#geno.name <- my.args[2]
#cvf.name <- my.args[3]

geno_pred(phenocsv = pheno.name,genocsv=geno.name, cvfcsv = cv.name, label=my_ph,mod = "BRR", phe =my_phe)



\end{lstlisting}

%$


\section{Results of GP} \label{AC:gp_res}
  
\begin{figure}[H]
  \centering \includegraphics[height=1.05\textheight, width=1.1\textwidth]{Figures/cor_plots_0}
  \decoRule
 \label{fig:bla}
\end{figure}

\begin{figure}[H]
  \centering \includegraphics[height=1.05\textheight, width=1.1\textwidth]{Figures/cor_plots_1}
  \decoRule
 \label{fig:bla}
\end{figure}

\begin{figure}[H]
  \centering \includegraphics[height=1.05\textheight, width=1.1\textwidth]{Figures/cor_plots_2}
  \decoRule
 \label{fig:bla}
\end{figure}

\begin{figure}[H]
  \centering \includegraphics[height=1.05\textheight, width=1.1\textwidth]{Figures/cor_plots_3}
  \decoRule
 \label{fig:bla}
\end{figure}

\begin{figure}[H]
  \centering \includegraphics[height=1.05\textheight, width=1.1\textwidth]{Figures/cor_plots_4}
  \decoRule
 \label{fig:bla}
\end{figure}

\begin{figure}[H]
  \centering \includegraphics[height=1.05\textheight, width=1.1\textwidth]{Figures/cor_plots_5}
  \decoRule
 \label{fig:bla}
\end{figure}

\begin{figure}[H]
  \centering \includegraphics[height=1.05\textheight, width=1.1\textwidth]{Figures/cor_plots_6}
  \decoRule
 \label{fig:bla}
\end{figure}

\begin{figure}[H]
  \centering \includegraphics[height=1.05\textheight, width=1.1\textwidth]{Figures/cor_plots_7}
  \decoRule
 \label{fig:bla}
\end{figure}

\begin{figure}[H]
  \centering \includegraphics[height=.35\textheight, width=.65\textwidth]{Figures/cor_plots_8}
  \decoRule
 \label{fig:bla}
\end{figure}
