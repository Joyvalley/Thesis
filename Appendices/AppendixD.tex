% Appendix Template

\chapter{Supplementary results} % Main appendix title

\label{AppendixD} % Change X to a consecutive letter; for referencing this appendix elsewhere, use \ref{AppendixX}

\section{Correlation plots of \textit{A. thaliana} GP}  \label{AC:gp_res}
  
\begin{figure}[H]
  \centering \includegraphics[height=1.05\textheight, width=1.1\textwidth]{Figures/cor_plots_0}
  \decoRule
 \label{fig:bla}
\end{figure}

\begin{figure}[H]
  \centering \includegraphics[height=1.05\textheight, width=1.1\textwidth]{Figures/cor_plots_1}
  \decoRule
 \label{fig:bla}
\end{figure}

\begin{figure}[H]
  \centering \includegraphics[height=1.05\textheight, width=1.1\textwidth]{Figures/cor_plots_2}
  \decoRule
 \label{fig:bla}
\end{figure}

\begin{figure}[H]
  \centering \includegraphics[height=1.05\textheight, width=1.1\textwidth]{Figures/cor_plots_3}
  \decoRule
 \label{fig:bla}
\end{figure}

\begin{figure}[H]
  \centering \includegraphics[height=1.05\textheight, width=1.1\textwidth]{Figures/cor_plots_4}
  \decoRule
 \label{fig:bla}
\end{figure}

\begin{figure}[H]
  \centering \includegraphics[height=1.05\textheight, width=1.1\textwidth]{Figures/cor_plots_5}
  \decoRule
 \label{fig:bla}
\end{figure}

\begin{figure}[H]
  \centering \includegraphics[height=1.05\textheight, width=1.1\textwidth]{Figures/cor_plots_6}
  \decoRule
 \label{fig:bla}
\end{figure}

\begin{figure}[H]
  \centering \includegraphics[height=1.05\textheight, width=1.1\textwidth]{Figures/cor_plots_7}
  \decoRule
 \label{fig:bla}
\end{figure}

\begin{figure}[H]
  \centering \includegraphics[height=.35\textheight, width=.65\textwidth]{Figures/cor_plots_8}
  \decoRule
 \label{fig:bla}
\end{figure}



\section{Haplotype structure of \textit{A. thaliana}} \label{haplo:str}

\begin{figure}[th]
\centering
\includegraphics[height=.55\textheight, width=1.1\textwidth]{Figures/chr1_hap}
\decoRule
\caption[Haplotype strutcture of chromosome 1 of \textit{A. thaliana}]{The number of segregating haplotypes with a polymorphism in at least one position over a stretch of 1 kBP.}
\label{fig:chr1}
\end{figure}



\begin{figure}[th]
\centering
\includegraphics[height=.55\textheight, width=1.1\textwidth]{Figures/chr2_hap}
\decoRule
\caption[Haplotype strutcture of chromosome 2 of \textit{A. thaliana}]{Number of segregating haplotypes with a polymorphism in at least one position over a stretch of 1 kBP. }
\label{fig:chr2}
\end{figure}


\begin{figure}[th]
\centering
\includegraphics[height=.55\textheight, width=1.1\textwidth]{Figures/chr3_hap}
\decoRule
\caption[Haplotype strutcture of chromosome 3 of \textit{A. thaliana}]{Number of segregating haplotypes with a polymorphism in at least one position over a stretch of 1 kBP. }
\label{fig:chr3}
\end{figure}


\begin{figure}[th]
\centering
\includegraphics[height=.55\textheight, width=1.1\textwidth]{Figures/chr4_hap}
\decoRule
\caption[Haplotype strutcture of chromosome 4 of \textit{A. thaliana}]{Number of segregating haplotypes with a polymorphism in at least one position over a stretch of 1 kBP. }
\label{fig:chr4}
\end{figure}


\begin{figure}[th]
\centering
\includegraphics[height=.55\textheight, width=1.1\textwidth]{Figures/chr5_hap}
\decoRule
\caption[Haplotype strutcture of chromosome 5 of \textit{A. thaliana}]{Number of segregating haplotypes with a polymorphism in at least one position over a stretch of 1 kBP. }
\label{fig:chr5}
\end{figure}


